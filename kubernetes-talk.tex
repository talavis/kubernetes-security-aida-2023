\documentclass{dcpresentation}

% Presentation info
\title{Kubernetes}
\subtitle{Security and stuff}
\author{Linus Östberg}
\newcommand{\authorEmail}{linus@oestberg.dev}
\institute{SciLifeLab Data Centre}
\date{}

\begin{document}
 \begin{frame}
  \maketitle
 \end{frame}
 
 % put logo in upper right corner to ape the official template
 \AddToShipoutPictureFG{
  \AtPageUpperRight{{\includegraphics[width=0.6cm,keepaspectratio]{img/scilifelab-symbol.pdf}}}
 }%

\begin{frame}
 \tableofcontents
\end{frame}

\section{Containers}
 
\begin{frame}{What is a container?}
 \begin{itemize}
  \item Running as a process on the host
  \item Namespaces
  \item Network, process (pid), filesystem (mount), (users)
  \item Cgroups (resources: CPU, RAM)
 \end{itemize}
\end{frame}

\begin{frame}{Containers and Virtual Machines}
 \begin{itemize}
  \item VM --- abstraction it believes it runs on its own computer
  \item Container --- looks like it's alone, but it is fully aware of its host (kernel)
 \end{itemize}
\end{frame}

\section{Kubernetes}

\begin{frame}
 \begin{itemize}
  \item Designed to be user-friendly
  \item By default it's less secure to start a container in Kubernetes than in Docker
 \end{itemize}
\end{frame}

\section{Security}

\subsection{Security Context}

\begin{frame}{root == root != root}
 \begin{itemize}
  \item Capabilities
  \item hostUsers
 \end{itemize}
\end{frame}

\begin{frame}{Privileged}
 \begin{itemize}
  \item 
 \end{itemize}
\end{frame}

\subsection{Seccomp, AppArmor, SELinux}

\begin{frame}{Permission Frameworks}
 \begin{frame}
  \item Limit what a container can do
 \end{frame}
\end{frame}

\subsection{Security Vulnerabilities}

\begin{frame}{Security Vulnerabilities}
 \begin{itemize}
  \item No known security vulnerabilities in images
  \item Scanning using tools
  \item Trivy etc
  \item Looks at apt database, package files (yarn.lock; requirements.txt ...)
 \end{itemize}  
\end{frame}

\begin{frame}{Security Vulnerabilities}{Malicious Compliance}
 \begin{itemize}
  \item \url{https://www.youtube.com/watch?v=9weGi0csBZM}
  \item Possible to trick the vulnerability scanners
  \item Remove package files
  \item Multi-stage builds
% Demo:
% Build data tracker production -> scan
% Scan yarn.lock
% Build data tracker build step -> scan
% (scan node:16-alpine) for reference
 \end{itemize}  
\end{frame}

\subsection{Secure Runtimes}

\begin{frame}
 \begin{itemize}
  \item Run containers using hardened runtimes
  \item Kata containers
  \item gVisor
 \end{itemize}
\end{frame}
 

\end{document}
