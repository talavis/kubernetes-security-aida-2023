\documentclass{dcpresentation}

% Presentation info
\title{Kubernetes}
\subtitle{Security and stuff}
\author{Linus Östberg}
\newcommand{\authorEmail}{linus@scilifelab.uu.se}
\institute{SciLifeLab Data Centre}
\date{}

\begin{document}
\begin{frame}
  \maketitle
\end{frame}

\begin{frame}{Who am I?}
 \begin{columns}
  \column{0.35\textwidth}
{\includegraphics[width=0.7\textwidth]{img/linus.png}}
  \column{0.65\textwidth}
  {\LARGE \scshape \bf Linus Östberg}
  
  {\small SciLifeLab Data Centre} \\
  {\tiny (soon Conoa AB)}

  \vspace{20pt}
  
  \begin{tabular}{cl}
   \faGithub & \href{https://github.com/talavis}{talavis} \\
   \faEnvelope & \href{mailto:linus@oestberg.dev}{linus@oestberg.dev}
  \end{tabular}
 \end{columns}
 
 \vspace{50pt}
 
 \begin{columns}
 \column{0.42\textwidth}
 \column{0.58\textwidth}
  \includegraphics[height=1.3cm]{img/ckad.png}
  \includegraphics[height=1.3cm]{img/cka.png}
  \includegraphics[height=1.3cm]{img/cks.png}
  \includegraphics[height=1.3cm]{img/cc.png}
  \end{columns}
\end{frame}

\section{Security Fundamentals}

\begin{frame}{Where to start?}
 \begin{itemize}
  \item Risk assessment
  \item Threat modelling
  \item Laws
  \item Requirements from organisations
 \end{itemize}
\end{frame}

\begin{frame}{CIA}
 \begin{itemize}
  \item Confidentiality
  \item Integrity
  \item Availability
 \end{itemize}
\end{frame}
% DAD - disclosure, alteration, destruction
% DREAD - Damage potential, reproducibility, exploitability, affected users, discoverability
% STRIDE - spoofing, tampering, repudiation, information disclosure, denial of service, elevation of priviledge

\begin{frame}{Fundamentals}
 \begin{itemize}
  \item Defence in depth
  \item Least privilege
  \item Zero trust
  \item Separation of duties
 \end{itemize}
\end{frame}

\begin{frame}{Cloud-native Security}
  \begin{itemize}
  \item Code
  \item Containers
  \item Clusters
  \item Cloud (on-premise)
  \end{itemize}
\end{frame}

\section{Code}

\begin{frame}{Code}
  \begin{itemize}
  \item Security starts in the application
  % Compromised app often means that all data for the app is exposed
  \item Secure software development
  \end{itemize}
\end{frame}

\begin{frame}{Supply Chain Security}
  \begin{itemize}
  \item Trusting third-party libraries
    \begin{itemize}
    \item Have you analysed the code of all third-party libraries before use?
    \item Did you verify their integrity?
    \end{itemize}
  \item Do you keep all libraries up-to-date?
  \end{itemize}
\end{frame}

\begin{frame}{Third-party security vulnerabilities}
  \begin{itemize}
  \item Automated warnings about security vulnerabilities
  \item Snyk
  \item Trivy
  \item Dependabot
  \end{itemize}
\end{frame}

\begin{frame}{TASK: Scanning with Trivy}
  \begin{itemize}
  \item \url{https://github.com/ScilifelabDataCentre/lunch-menu}
  \item Are there any known vulnerabilities?
  \item Use \texttt{trivy}, \texttt{snyk}, or any other scanner
  \item Hints:
    \begin{itemize}
    \item \texttt{requirements.txt}
    \item \texttt{yarn.lock}
    \item \texttt{trivy fs}
    \end{itemize}
  \end{itemize}
\end{frame}

\section{Container}

\begin{frame}
  \centering{ \alert{ \huge What is a container?} }
\end{frame}

\begin{frame}
  \centering{ \alert{ \Large What are the differences between a container and a virtual machine?} }
\end{frame}

\begin{frame}{Containers}
  \begin{itemize}
  \item Namespaced processes
  \item OCI
    \begin{itemize}
    \item Image
    \item Runtime
    \item Distribution
    \end{itemize}
  \end{itemize}
\end{frame}

% Running namespaced processes on a host
% Network, process (pid), filesystem (mount), (users), UTS (domain/hostname), IPC (inter-process communiation (shared memory etc)
% Cgroups (resources: CPU, RAM)
% lsns


\begin{frame}{Container Runtimes}
  \begin{itemize}
  \item Docker
  \item Containerd
  \item CRI-O
  \item Gvisor
  \item Kata
  \item Firecracker
  \end{itemize}
\end{frame}

\begin{frame}
  \begin{itemize}
  \item Containers share the kernel
  \item \alert<2>{Very sensitive data should be in different vms/clusters}
  \end{itemize}
\end{frame}

\begin{frame}{root == root != root}
  \begin{itemize}
  \item Container user == host user
    \begin{itemize}
    \item User namespaces exist, but have limited support
    \end{itemize}
  \item Capabilities
    % demo: iptables
  \item hostUsers
  \item Privileged container == danger
  \end{itemize}
\end{frame}

% DEMO: 
% docker run -it --rm nginx bash
% docker run -it --cap-add=NET_ADMIN --rm nginx bash
% docker run -it --privileged --rm nginx bash

% DEMO:
% docker run -it -v /:/host:rw --rm nginx bash
% touch asd /host/Applications
% docker run -it -u 10000 -v /:/host:rw --rm nginx bash
% touch asd /host/Applications

\begin{frame}
  \begin{itemize}
  \item Run containers using hardened runtimes
  \item Kata containers (virtualisation)
  \item gVisor (sandbox)
  \end{itemize}
\end{frame}

\begin{frame}{TASK: use gvisor}
 \begin{itemize}
  \item \url{https://killercoda.com/killer-shell-cks/scenario/sandbox-gvisor}
 \end{itemize}
\end{frame}

% TASK: https://killercoda.com/killer-shell-cks/scenario/sandbox-gvisor

\begin{frame}{Container drift}
 \begin{itemize}
  \item Preventing containers from doing unintended actions
 \end{itemize}
\end{frame}

\begin{frame}{Seccomp, Apparmor, SELinux}
  \begin{itemize}
  \item Security frameworks to add extra security
  \item Ubuntu: seccomp, apparmor
  \item Red Hat: seccomp, selinux
  \item Define a security profile to be used with a container
  \item Optimal security: define a specialised profile for each container
  \item Using the "runtime default" is better than nothing
  \end{itemize}
\end{frame}

\begin{frame}{TASK: using Apparmor}
 \begin{itemize}
  \item \url{https://killercoda.com/killer-shell-cks/scenario/apparmor}
 \end{itemize}
\end{frame}


\begin{frame}{Security Vulnerabilities}
  \begin{itemize}
  \item Scan the packaged software in container images
  \item E.g. Trivy
  \end{itemize}
\end{frame}


\begin{frame}{TASK: image and code scanning}
  \begin{itemize}
  \item \url{https://github.com/ScilifelabDataCentre/lunch-menu}
  \item Do the latest container images contain any known vulnerabilities?
  \item Use Trivy or any other scanner
  \item \texttt{trivy image container:label}
  \end{itemize}
\end{frame}

% something off 
% - why do we have fewer issues now than in the earlier fs scan?
% - why is there a security issue in backend?
% check the Dockerfile

\begin{frame}{Malicious Compliance}
  \begin{itemize}
  \item \url{https://www.youtube.com/watch?v=9weGi0csBZM}
  \item Possible to trick the vulnerability scanners
  \begin{itemize}
  \item Remove package management files
  \item Symlinks
  \item Multi-stage builds
  \end{itemize} 
  \end{itemize}  
\end{frame}

% scanners:
% package database
% yarn.lock, requirements.txt etc
% standard installation paths etc
% cim

\begin{frame}{Building containers}
 \begin{itemize}
  \item Minimise
  \begin{itemize}
   \item Minimal base % alpine, scratch, distroless
   \item Remove unused binaries
   \item Squash layers % slim, dive
  \end{itemize}
  \item Never include secrets during the build steps
  \item Automate
  \item Do not run as root
  \item Immutable
 \end{itemize}
\end{frame}

% DEMO: improving a Dockerfile
% https://github.com/ScilifelabDataCentre/lunch-menu/tree/dockerfile-bad

\begin{frame}{Kubernetes}
  \begin{itemize}
  \item Container orchestration
  \item Designed to be flexible
  \item Declare wanted state
    \begin{itemize}
    \item Reconciliation loop
    \end{itemize}
  \end{itemize}
\end{frame}

% starting in k8s by default less secure than starting in e.g. docker 

\begin{frame}{Running With Scissors}
  \url{https://www.youtube.com/watch?v=ltrV-Qmh3oY}
% going from a "minimal" user to full root rights on all nodes
\end{frame}

% service account with pod * permissions
% extract token
% start privileged ubuntu pod
% host filesystem as well?
% nsenter -t 1 -a bash
% create invisible pod (non-existing namespace static pod)

\begin{frame}{OWASP Kubernetes Top Ten}
  {\url{https://owasp.org/www-project-kubernetes-top-ten/}}
  \begin{description}
  \item[K01] Insecure Workload Configurations
  \item[K02] Supply Chain Vulnerabilities
  \item[K03] Overly Permissive RBAC Configurations
  \item[K04] Lack of Centralized Policy Enforcement
  \item[K05] Inadequate Logging and Monitoring
  \item[K06] Broken Authentication Mechanisms
  \item[K07] Missing Network Segmentation Controls
  \item[K08] Secrets Management Failures
  \item[K09] Misconfigured Cluster Components
  \item[K10] Outdated and Vulnerable Kubernetes Components
  \end{description}
\end{frame}

\begin{frame}{K10: Outdated and Vulnerable Kubernetes Components}
  \begin{itemize}
  \item Keep Kubernetes updated
  \item Three most recent minor releases:
    \begin{itemize}
    \item 1.28
    \item 1.27
    \item 1.26
    \end{itemize}
  \item $\sim$1 year support
  \item Distributions may be supported longer
  \end{itemize}
\end{frame}

\begin{frame}{K09: Misconfigured Cluster Components}
  \begin{itemize}
  \item CIS Benchmarks (kube-bench)
    % show examples from a CIS benchmark
  \end{itemize}
\end{frame}

\begin{frame}{TASK: Using kube-bench}
 \begin{itemize}
  \url{https://killercoda.com/killer-shell-cks/scenario/cis-benchmarks-kube-bench-fix-controlplane}
 \end{itemize}
\end{frame}
%TASK: run kube-bench
% https://killercoda.com/killer-shell-cks/scenario/cis-benchmarks-kube-bench-fix-controlplane

\begin{frame}{K08: Secrets Management Failures}
  \begin{itemize}
  \item Encrypt secrets
  \item Vault
  \item Sealed secrets etc
  \end{itemize}
\end{frame}

\begin{frame}{TASK: encrypting ETCD}
 \begin{itemize}
  \url{https://killercoda.com/killer-shell-cks/scenario/secret-etcd-encryption}
 \end{itemize}
\end{frame}

\begin{frame}{K07: Missing Network Segmentation Controls}
  \begin{itemize}
  \item Zero trust 
  \item Network policies
  \end{itemize}
\end{frame}

\begin{frame}{Network Policies}
  \begin{itemize}
  \item \url{https://editor.networkpolicy.io/}
  \item Default: deny-all for namespace
  \item Minimise access
  \end{itemize}
  
\begin{frame}{TASK: Creating network policies}
 \begin{itemize}
   \item 
 \end{itemize}
\end{frame}  
  
  % TASK: make a network policy
  % Form manager? Menu page?
\end{frame}

\begin{frame}{K06: Broken Authentication Mechanisms}
  \begin{itemize}
  \item certificate = forever
  \item Service account tokens
  \item use MFA if possible
  \end{itemize}
\end{frame}

% DEMO: rancher - github

\begin{frame}{K05: Inadequate Logging and Monitoring}
  \begin{itemize}
  \item Audit logs, logs in external system
  \item Falco etc for monitoring
  \item Fluentbit etc
  \item Loki
  \end{itemize}
\end{frame}

\begin{frame}{K04: Lack of Centralized Policy Enforcement}
  \begin{itemize}
  \item OPA Gatekeeper, Kyverno
  \item ``Policies in git''
  \end{itemize}
\end{frame}

\begin{frame}{TASK: Using Kyverno}
 https://killercoda.com/kyverno/scenario/intro
\end{frame}

\begin{frame}
  \begin{itemize}
  \item GitOps --- Argo
  \item Infrastructure as code
  \item Ansible
  \item ``Replace one node every month''
  \end{itemize}
\end{frame}

% DEMO: Argo

\begin{frame}{K03: Overly Permissive RBAC Configurations}
  \begin{itemize}
  \item Least Privilege
  \item Service account and user RBAC permissions
  \item Limit use of ClusterRoleBinding
  \item Not everyone needs admin permissions
  \end{itemize}
\end{frame}

\begin{frame}{K02: Supply Chain Vulnerabilities}
  \begin{itemize}
  \item Security vulnerabilities in third-party libraries
  \item Insecure/malicious images
  \item Container signing
  \item Image minimisation
  \end{itemize}
\end{frame}

\begin{frame}{Supply Chain Vulnerabilities}
  \begin{itemize}
  \item Signing
  \item Do not use untrusted images
  \end{itemize}
  Signing

\end{frame}


\begin{frame}{K01: Insecure Workload Configurations}
  \begin{itemize}
  \item Scanners: kube-score, kubesec, kubeaudit, snyk ...
  \item Pod Security Standards
  \item Pod admission: kyverno, Open Policy Agent Gatekeeper
  \end{itemize}
\end{frame}

\begin{frame}{Pod Security Standards}
  \begin{itemize}
  \item \url{https://kubernetes.io/docs/concepts/security/pod-security-standards/}
  \item Apply to a namespace
  \item \texttt{pod-security.kubernetes.io/<MODE>: <LEVEL>}
  \item \texttt{pod-security.kubernetes.io/<MODE>-version: <VERSION>}
  \item Enforce, audit, warn
  \end{itemize}

\end{frame}


% TASK: Make deployment follow Pod security standard - restricted 

\begin{frame}{TASK: Create a hardened deployment}
  \begin{itemize}
  \item Make a deployment of \texttt{ghcr.io/scilifelabdatacentre/menu-backend:latest}
  \item Create a new namespace and apply the pod security standard restricted to it
  \item Update your deployment to allow deployment in the created namespace
  \end{itemize}
\end{frame}


\begin{frame}{Host security}
  \begin{itemize}
  \item Hardening
  \item CIS Benchmark
  \item Minimisation
  \item Firewalls
  \end{itemize}
\end{frame}


\begin{frame}{Compliance}
  \begin{itemize}
  \item Not just a checklist
  \item Aid to make your systems more secure 
  \end{itemize}
\end{frame}

\begin{frame}{Gitops}

\end{frame}

\begin{frame}{Enforcement}
% pod configuration

\end{frame}

\begin{frame}
  \begin{itemize}
  \item Least Privilege
  \item Defence in depth (layered security)
  \item Zero Trust
  \end{itemize}
\end{frame}

\end{document}
