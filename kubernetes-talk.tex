\documentclass{dcpresentation}

% Presentation info
\title{Kubernetes}
\subtitle{Security and stuff}
\author{Linus Östberg}
\newcommand{\authorEmail}{linus@scilifelab.uu.se}
\institute{SciLifeLab Data Centre}
\date{}

\begin{document}
\begin{frame}
  \maketitle
\end{frame}

\begin{frame}{Who am I?}
 \begin{columns}
  \column{0.35\textwidth}
{\includegraphics[width=0.7\textwidth]{img/linus.png}}
  \column{0.65\textwidth}
  {\LARGE \scshape \bf Linus Östberg}
  
  {\small SciLifeLab Data Centre} \\
  {\tiny (soon Conoa AB)}

  \vspace{20pt}
  
  \begin{tabular}{cl}
   \faGithub & \href{https://github.com/talavis}{talavis} \\
   \faEnvelope & \href{mailto:linus@oestberg.dev}{linus@oestberg.dev}
  \end{tabular}
 \end{columns}
 
 \vspace{50pt}
 
 \begin{columns}
 \column{0.35\textwidth}
 \column{0.65\textwidth}
  \includegraphics[height=1.3cm]{img/kcna.png}
  \includegraphics[height=1.3cm]{img/ckad.png}
  \includegraphics[height=1.3cm]{img/cka.png}
  \includegraphics[height=1.3cm]{img/cks.png}
  \includegraphics[height=1.3cm]{img/cc.png}
  \end{columns}
\end{frame}

\section{Security Fundamentals}

\begin{frame}{Where to start?}
 \begin{itemize}
  \item Risk assessment
  \item Threat modelling
  \item Laws, regulations
 \end{itemize}
\end{frame}

\begin{frame}{CIA}
 \begin{itemize}
  \item Confidentiality
  \item Integrity
  \item Availability
 \end{itemize}
\end{frame}

\begin{frame}{Fundamentals}
 \begin{itemize}
  \item Defence in depth
  \item Least privilege
  \item Zero trust
 \end{itemize}
\end{frame}

\begin{frame}{Cloud-native Security}
  \begin{itemize}
  \item Code
  \item Containers
  \item Clusters
  \item Cloud (on-premise)
  \end{itemize}
\end{frame}

\section{Code}

\begin{frame}{Code}
  \begin{itemize}
  \item Security starts in the application
  \item Secure software development
  \item Usable by default, not secure
  \end{itemize}
\end{frame}

\begin{frame}{Supply Chain Security}
  \begin{itemize}
  \item Trusting third-party libraries
    \begin{itemize}
    \item Code evaluation?
    \item Integrity check?
    \end{itemize}
  \item Up-to-date?
  \end{itemize}
\end{frame}

\begin{frame}{Third-party security vulnerabilities}
  \begin{itemize}
  \item Vulnerability scanning
  \item Snyk
  \item Trivy
  \item Dependabot
  \end{itemize}
\end{frame}

\begin{frame}{TASK: Scanning with Trivy}
  \begin{itemize}
  \item \url{https://github.com/ScilifelabDataCentre/lunch-menu}
  \item Are there any known vulnerabilities?
  \item Use \texttt{trivy}, \texttt{snyk}, or any other scanner
  \item Hints:
    \begin{itemize}
    \item \texttt{requirements.txt}
    \item \texttt{yarn.lock}
    \item \texttt{trivy fs}
    \end{itemize}
  \end{itemize}
\end{frame}

\section{Container}

\begin{frame}
  \centering{ \alert{ \huge What is a container?} }
\end{frame}

% Running namespaced processes on a host
% Network, process (pid), filesystem (mount), (users), UTS (domain/hostname), IPC (inter-process communiation (shared memory etc)
% Cgroups (resources: CPU, RAM)
% lsns

\begin{frame}{Open Container Initiative}
  \begin{itemize}
  \item OCI
    \begin{itemize}
    \item Image
    \item Runtime
    \item Distribution
    \end{itemize}
  \end{itemize}
\end{frame}

\begin{frame}{Container Runtimes}
  \begin{itemize}
  \item Docker
  \item Containerd
  \item CRI-O
  \item Gvisor
  \item Kata
  \item Firecracker
  \end{itemize}
\end{frame}

\begin{frame}
  \centering{ \alert{ \Large What are the differences between a container and a virtual machine?} }
\end{frame}

\begin{frame}
  \begin{itemize}
  \item Containers share the kernel
  \item \alert{Very sensitive data should be in different vms/clusters}
  \end{itemize}
\end{frame}

\begin{frame}{root == root != root}
  \begin{itemize}
  \item Container user == host user
    \begin{itemize}
    \item User namespaces exist, but have limited support
    \end{itemize}
  \item Capabilities
    % demo: iptables
  \item hostUsers
  \item Privileged container == danger
  \item<2> \alert{Do not run your containers as root}
  \end{itemize}
\end{frame}

% DEMO: 
% docker run -it --rm nginx bash
% docker run -it --cap-add=NET_ADMIN --rm nginx bash
% docker run -it --privileged --rm nginx bash

% DEMO:
% docker run -it -v /:/host:rw --rm nginx bash
% touch asd /host/Applications
% docker run -it -u 10000 -v /:/host:rw --rm nginx bash
% touch asd /host/Applications

\begin{frame}{Hardened container runtimes}
  \begin{itemize}
  \item Run containers using hardened runtimes
  \item Kata containers (virtualisation)
  \item gVisor (sandbox)
  \end{itemize}
\end{frame}

\begin{frame}{TASK: use gvisor}
 \begin{itemize}
  \item RuntimeClass
  \item \url{https://killercoda.com/killer-shell-cks/scenario/sandbox-gvisor}
 \end{itemize}
\end{frame}

% TASK: https://killercoda.com/killer-shell-cks/scenario/sandbox-gvisor

\begin{frame}{Container Drift}
 \begin{itemize}
  \item Containers are often intended to do only one thing
  \item Containers that do other things are "drifting"
  \item Preventing drift may improve security
 \end{itemize}
\end{frame}

\begin{frame}{Seccomp, Apparmor, SELinux}
  \begin{itemize}
  \item Security frameworks to add extra security
  \item Ubuntu: seccomp, apparmor
  \item Red Hat: seccomp, selinux
  \item Define a security profile to be used with a container
  \item Optimal security: define a specialised profile for each container
  \item Using the "runtime default" is better than nothing
  \end{itemize}
\end{frame}

\begin{frame}{TASK: using Apparmor}
 \begin{itemize}
  \item \texttt{\scriptsize container.apparmor.security.beta.kubernetes.io/<container\_name>: <profile\_ref>}
  \item \texttt{runtime/default}
  \item \texttt{localhost/k8s-apparmor-deny-everything}
  \item Create profile, make sure it exists on all nodes where the container may run
  \item \url{https://killercoda.com/killer-shell-cks/scenario/apparmor}
 \end{itemize}
\end{frame}

\begin{frame}{Supply Chain Security}
  \begin{itemize}
  \item Security vulnerabilities in third-party libraries
  \item Insecure/malicious images
  \item Container signing
  \item Image minimisation
  \end{itemize}
\end{frame}

\begin{frame}{Security Vulnerabilities}
  \begin{itemize}
  \item Scan the container images
  \item Discover issues in the application and helper files
  \item E.g. Trivy
  \end{itemize}
\end{frame}


\begin{frame}{TASK: image and code scanning}
  \begin{itemize}
  \item \url{https://github.com/ScilifelabDataCentre/lunch-menu}
  \item Do the latest container images contain any known vulnerabilities?
  \item Does the containers with tag 23.7.2 contain any known vulnerabilities?
  \item Use Trivy or any other scanner
  \item \texttt{trivy image container:label}
  \item Does this make sense?
  \end{itemize}
\end{frame}


\begin{frame}{Malicious Compliance}
  \begin{itemize}
  \item \url{https://www.youtube.com/watch?v=9weGi0csBZM}
  \item Possible to trick the vulnerability scanners
  \begin{itemize}
  \item Remove package management files
  \item Symlinks
  \item Multi-stage builds
  \end{itemize} 
  \end{itemize}  
\end{frame}

% scanners:
% package database
% yarn.lock, requirements.txt etc
% standard installation paths etc
% cim

\begin{frame}{Building containers}
 \begin{itemize}
  \item Minimise
  \begin{itemize}
   \item Minimal base % alpine, scratch, distroless
   \item Remove unused binaries
   \item Squash layers % slim, dive
  \end{itemize}
  \item Never include secrets during the build steps
  \item Automate
  \item Do not run as root
  \item Immutable
 \end{itemize}
\end{frame}


\section{Kubernetes}


\begin{frame}{Running With Scissors}
  \url{https://www.youtube.com/watch?v=ltrV-Qmh3oY}
% going from a "minimal" user to full root rights on all nodes
\end{frame}


\begin{frame}{Kubernetes}
  \begin{itemize}
  \item Container orchestration
  \item Designed to be flexible
  \item Declare wanted state
    \begin{itemize}
    \item Reconciliation loop
    \end{itemize}
  \end{itemize}
\end{frame}


\begin{frame}{OWASP Kubernetes Top Ten}
  {\url{https://owasp.org/www-project-kubernetes-top-ten/}}
  \begin{description}
  \item[K01] Insecure Workload Configurations
  \item[K02] Supply Chain Vulnerabilities
  \item[K03] Overly Permissive RBAC Configurations
  \item[K04] Lack of Centralized Policy Enforcement
  \item[K05] Inadequate Logging and Monitoring
  \item[K06] Broken Authentication Mechanisms
  \item[K07] Missing Network Segmentation Controls
  \item[K08] Secrets Management Failures
  \item[K09] Misconfigured Cluster Components
  \item[K10] Outdated and Vulnerable Kubernetes Components
  \end{description}
\end{frame}


\begin{frame}{K10: Outdated and Vulnerable Kubernetes Components}
  \begin{itemize}
  \item Keep Kubernetes updated
  \item Currently supported releases:
    \begin{itemize}
    \item 1.28
    \item 1.27
    \item 1.26
    \item 1.25
    \end{itemize}
  \item $\sim$1 year support
  \item Distributions may be supported longer
  \end{itemize}
\end{frame}


\begin{frame}{K09: Misconfigured Cluster Components}
  \begin{itemize}
  \item CIS Benchmarks (kube-bench)
    % show examples from a CIS benchmark
  \end{itemize}
\end{frame}


\begin{frame}{TASK: Using kube-bench}
 \begin{itemize}
  \item Using kube-bench
  \item \url{https://killercoda.com/killer-shell-cks/scenario/cis-benchmarks-kube-bench-fix-controlplane}
 \end{itemize}
\end{frame}


\begin{frame}{K08: Secrets Management Failures}
  \begin{itemize}
  \item Encrypt secrets
  \item Vault
  \item Sealed secrets etc
  \end{itemize}
\end{frame}


\begin{frame}{K07: Missing Network Segmentation Controls}
  \begin{itemize}
  \item Zero trust 
  \item Network policies
  \end{itemize}
\end{frame}

\begin{frame}{Network Policies}
  \begin{itemize}
  \item \url{https://editor.networkpolicy.io/}
  \item Default: deny-all for namespace
  \item Minimise access
  \end{itemize}
\end{frame}
 
\begin{frame}{TASK: Creating network policies}
 \begin{itemize}
   \item Practice creating network policies
   \item \url{https://killercoda.com/killer-shell-cks/scenario/networkpolicy-namespace-communication}
   \item \url{https://editor.networkpolicy.io/}
 \end{itemize}
\end{frame}


\begin{frame}{K06: Broken Authentication Mechanisms}
  \begin{itemize}
    \item Certificates last until expiration
    \item Service account tokens
    \item Use MFA if possible
    \item Be careful with service account tokens
  \end{itemize}
\end{frame}

\begin{frame}{K05: Inadequate Logging and Monitoring}
  \begin{itemize}
  \item Save logs in an external system
  \begin{itemize}
   \item Kubernetes Audit logs
   \item Application/container logs
   \item Event logs
   \item Operating system logs
   \item Network logs
  \end{itemize}
  \item \textbf{Monitor the logs}
  \end{itemize}
\end{frame}

\begin{frame}{K04: Lack of Centralized Policy Enforcement}
  \begin{itemize}
  \item Policies about what may run on the cluster
  \item Policies as Code --- PaC
  \item Pod Security Standards
  \item OPA Gatekeeper
  \item Kyverno
  \end{itemize}
\end{frame}

\begin{frame}{Pod Security Standards}
  \begin{itemize}
  \item \url{https://kubernetes.io/docs/concepts/security/pod-security-standards/}
  \item Apply to a namespace
  \item \texttt{pod-security.kubernetes.io/<MODE>: <LEVEL>}
  \item \texttt{pod-security.kubernetes.io/<MODE>-version: <VERSION>}
  \item Enforce, audit, warn
  \end{itemize}
\end{frame}

\begin{frame}{OPA Gatekeeper}
 \begin{itemize}
  \item Open Policy Agent
  \item Rego
  \item OPA Gatekeeper
  \item Contstraint templates
  \item Constraints
  \item \url{https://killercoda.com/opa/scenario/intro}
 \end{itemize}
\end{frame}


\begin{frame}{Kyverno}
 \begin{itemize}
  \item YAML
  \item Kubernetes-native
  \url{https://killercoda.com/kyverno/scenario/intro}
 \end{itemize}
\end{frame}


\begin{frame}{K03: Overly Permissive RBAC Configurations}
  \begin{itemize}
  \item Least Privilege
  \item Service account and user RBAC permissions
  \item Limit use of ClusterRoleBinding
  \item Not everyone needs admin permissions
  \item Be careful with service account tokens
  \end{itemize}
\end{frame}

\begin{frame}{K02: Supply Chain Vulnerabilities}
  \begin{itemize}
  \item Security vulnerabilities in third-party libraries
  \item Insecure/malicious images
  \begin{itemize}
   \item Do not use untrusted images
  \end{itemize}
  \item Secure the CI/CD pipelines
  \item Software bill of materials (SBOM)
  \item Container signing
  \end{itemize}
\end{frame}


\begin{frame}{K01: Insecure Workload Configurations}
  \begin{itemize}
  \item A container in Kubernetes is by default less secure than in Docker
  \item Need to improve the configuration
  \item Tools:
  \begin{itemize}
   \item kube-score
   \item kubesec
   \item snyk
  \end{itemize}
 \end{itemize}
\end{frame}


\begin{frame}{No root}
  \begin{itemize}
    \item securityContext
    \begin{itemize}
      \item runAsUser
      \item runAsGroup
      \item allowPrivilegeEscalation
      \item privileged
      \item runAsNonRoot
      \item capabilities
    \end{itemize}
  \end{itemize}
\end{frame}

\begin{frame}{Service account tokens}
  \begin{itemize}
    \item Never mount service account tokens unless they are needed
    \item \texttt{automountServiceAccountToken: false}
  \end{itemize}
\end{frame}

\begin{frame}{Apparmor, Seccomp, SELinux}
  \begin{itemize}
    \item Use Seccomp, SELinux, and Apparmor
    \begin{itemize}
     \item Must be supported by the hosts
    \end{itemize}
  \end{itemize}
\end{frame}

\begin{frame}[fragile]{Seccomp and SELinux}
\begin{verbatim}
 securityContext:
  seccompProfile:
    type: RuntimeDefault
  seLinuxOptions:
    level: "s0:c123,c456"
\end{verbatim}
\end{frame}

\begin{frame}[fragile]{Apparmor}
\begin{verbatim}
spec:
  template:
    metadata:
      annotations:
        container.apparmor.security.beta.kubernetes.io/containername: runtime/default
\end{verbatim}
\end{frame}



\begin{frame}{Immutability}
  \begin{itemize}
    \item A container should be immutable
    \item \texttt{readOnlyRootFilesystem: true}
    \item Binaries can be run without being saved to disk
    \item Add emptyDirs if specific folders need writing 
  \end{itemize}
\end{frame}


\begin{frame}{Enforcement}
\begin{itemize}
 \item Pre-deployment:
 \begin{itemize}
  \item kube-score
  \item kubesec
  \item kubeaudit
  \item snyk
 \end{itemize}
 \item Admission:
 \begin{itemize}
  \item Pod security standards
  \item Kyverno
  \item OPA Gatekeeper
 \end{itemize}
\end{itemize}
\end{frame}

\begin{frame}{TASK: Create a hardened deployment}
  \begin{itemize}
  \item Make a deployment of \texttt{ghcr.io/scilifelabdatacentre/menu-backend:latest}
  \item Create a new namespace and apply the pod security standard restricted to it
  \item Update your deployment to allow deployment in the created namespace
  \item Optional: add network policies or any other relevant protections
  \end{itemize}
\end{frame}


\section{Cloud}

\begin{frame}{Host security}
  \begin{itemize}
  \item Hardening
  \item CIS Benchmark
  \item Minimisation
  \item Firewalls
  \end{itemize}
\end{frame}

\section{Final Thoughts}

\begin{frame}{GitOps}
 \begin{itemize}
  \item Using a Git repository as the source of truth
  \item Any changes are committed to Git
  \item Current cluster state always saved to Git
  \item Argo, Flux
  \item Analysis tools can be run as part of CI
  \item Adds non-repudiation to any changes
 \end{itemize}
\end{frame}

\begin{frame}{Compliance}
  \begin{itemize}
  \item Not just a checklist
  \item Aid to make your systems more secure 
  \end{itemize}
\end{frame}

\begin{frame}
  \begin{itemize}
  \item Least Privilege
  \item Defence in depth (layered security)
  \item Zero Trust
  \end{itemize}
\end{frame}

\begin{frame}
 \begin{center}
  \LARGE Keep on learning!
 \end{center}  
\end{frame}

\begin{frame}
 \begin{center}
  \LARGE Thank you for listening!
  \vspace{20pt}
  \Large Questions?
 \end{center}  
\end{frame}

\end{document}
