\documentclass{dcpresentation}

% Presentation info
\title{Kubernetes}
\subtitle{Security and stuff}
\author{Linus Östberg}
\newcommand{\authorEmail}{linus@oestberg.dev}
\institute{SciLifeLab Data Centre}
\date{}

\begin{document}
 \begin{frame}
  \maketitle
 \end{frame}

\begin{frame}
 \centering{ \alert{ \huge What is a container?} }
\end{frame}
% Running namespaced processes on a host
% Network, process (pid), filesystem (mount), (users)
% Cgroups (resources: CPU, RAM)

\begin{frame}
 \centering{ \alert{ \Large What is the difference between a container and a virtual machine?} }
\end{frame}


% \begin{frame}{Containers and Virtual Machines}
%  \begin{itemize}
%   \item VM --- abstraction it believes it runs on its own computer
%   \item Container --- looks like it's alone, but it is fully aware of its host (kernel)
%  \end{itemize}
% \end{frame}

\begin{frame}
 \begin{itemize}
  \item Containers are not fully isolated
  \item Taking control of the host means that you have control over all containers
 \end{itemize}
\end{frame}

\begin{frame}{Kubernetes}
 \begin{itemize}
  \item Container orchestration
  \item Designed to be flexible
  \item Declare wanted state
  \begin{itemize}
   \item Reconciliation loop
  \end{itemize}
 \end{itemize}
\end{frame}


\begin{frame}
 \begin{itemize}
  \item Least Privilege
  \item Defence in depth (layered security)
  \item Zero Trust
 \end{itemize}
\end{frame}

% starting in k8s by default less secure than starting in e.g. docker 

\begin{frame}{OWASP Kubernetes Top Ten}
  {\url{https://owasp.org/www-project-kubernetes-top-ten/}}
 \begin{description}
  \item[K01] Insecure Workload Configurations
  \item[K02] Supply Chain Vulnerabilities
  \item[K03] Overly Permissive RBAC Configurations
  \item[K04] Lack of Centralized Policy Enforcement
  \item[K05] Inadequate Logging and Monitoring
  \item[K06] Broken Authentication Mechanisms
  \item[K07] Missing Network Segmentation Controls
  \item[K08] Secrets Management Failures
  \item[K09] Misconfigured Cluster Components
  \item[K10] Outdated and Vulnerable Kubernetes Components
 \end{description}
\end{frame}

\begin{frame}{K10: Outdated and Vulnerable Kubernetes Components}
 \begin{itemize}
  \item Keep Kubernetes updated
  \item Three most recent minor releases:
  \begin{itemize}
   \item 1.28
   \item 1.27
   \item 1.26
  \end{itemize}
  \item $\sim$1 year support
  \item Distributions may be supported longer
 \end{itemize}
\end{frame}

\begin{frame}{K09: Misconfigured Cluster Components}
 \begin{itemize}
  \item CIS Benchmarks (kube-bench)
  % show examples from a CIS benchmark
 \end{itemize}
\end{frame}
%TASK: run kube-bench
% https://killercoda.com/killer-shell-cks/scenario/cis-benchmarks-kube-bench-fix-controlplane

\begin{frame}{K08: Secrets Management Failures}
 \begin{itemize}
  \item Encrypt secrets
  \item Vault
  \item Sealed secrets etc
 \end{itemize}
\end{frame}

% TASK: look at secrets in etcd: encrypted?

\begin{frame}{K07: Missing Network Segmentation Controls}
 \begin{itemize}
  \item Zero trust 
  \item Network policies
 \end{itemize}
\end{frame}

\begin{frame}{Network Policies}
 \begin{itemize}
  \item \url{https://editor.networkpolicy.io/}
  \item Default: deny-all for namespace
  \item Minimise access
 \end{itemize}
 
 % TASK: make a network policy
 % Form manager? Menu page?
\end{frame}

\begin{frame}{K06: Broken Authentication Mechanisms}
 \begin{itemize}
  \item certificate = forever
  \item Service account tokens
  \item use MFA if possible
 \end{itemize}
\end{frame}

% DEMO: rancher - github

\begin{frame}{K05: Inadequate Logging and Monitoring}
 \begin{itemize}
  \item Audit logs, logs in external system
  \item Falco etc for monitoring
  \item Fluentbit etc
  \item Loki
 \end{itemize}
\end{frame}

\begin{frame}{K04: Lack of Centralized Policy Enforcement}
 \begin{itemize}
  \item OPA Gatekeeper, Kyverno
  \item ``Policies in git''
 \end{itemize}
\end{frame}

\begin{frame}
 \begin{itemize}
  \item GitOps --- Argo
  \item Infrastructure as code
  \item Ansible
  \item ``Replace one node every month''
 \end{itemize}
\end{frame}

% DEMO: Argo

\begin{frame}{K03: Overly Permissive RBAC Configurations}
 \begin{itemize}
  \item Least Privilege
  \item Service account and user RBAC permissions
  \item Limit use of ClusterRoleBinding
  \item Not everyone needs admin permissions
 \end{itemize}
\end{frame}

\begin{frame}{K02: Supply Chain Vulnerabilities}
 \begin{itemize}
  \item Security vulnerabilities in third-party libraries
  \item Insecure/malicious images
  \item Container signing
  \item Image minimisation
 \end{itemize}
\end{frame}

\begin{frame}{Supply Chain Vulnerabilities}
 \begin{itemize}
  \item Signing
  \item Do not use untrusted images
 \end{itemize}
Signing

\end{frame}

\begin{frame}{Security Vulnerabilities}
 \begin{itemize}
  \item No known security vulnerabilities in images
  \item Scanning using tools
  \item Trivy etc
  \item Looks at apt database, package files (yarn.lock; requirements.txt ...)
 \end{itemize}  
\end{frame}


% TASK: build and scan image
% vulnerable node modules
% no warning in image
% https://github.com/ScilifelabDataCentre/lunch-menu
% Latest deployed frontend: any security issues?
% 

\begin{frame}{TASK: image and code scanning}
 \begin{itemize}
  \item \url{https://github.com/ScilifelabDataCentre/lunch-menu}
  \item Does menu-frontend:latest contain any known vulnerabilities?
  \item Use Trivy or any other relevant tools
  \item \texttt{trivy image container:label}
 \end{itemize}
\end{frame}


\begin{frame}{Security Vulnerabilities}{Malicious Compliance}
 \begin{itemize}
  \item \url{https://www.youtube.com/watch?v=9weGi0csBZM}
  \item Possible to trick the vulnerability scanners
  \item Remove package management files
  \item Symlinks
  \item Multi-stage builds
 \end{itemize}  
\end{frame}

% scanners:
% package database
% yarn.lock, requirements.txt etc
% standard installation paths etc
% cim

\begin{frame}{K01: Insecure Workload Configurations}
 \begin{itemize}
  \item Scanners: kube-score, kubesec, kubeaudit, snyk ...
  \item Pod Security Standards
  \item Pod admission: kyverno, Open Policy Agent Gatekeeper
 \end{itemize}
\end{frame}

\begin{frame}{Pod Security Standards}
 \begin{itemize}
  \item \url{https://kubernetes.io/docs/concepts/security/pod-security-standards/}
  \item Apply to a namespace
  \item \texttt{pod-security.kubernetes.io/<MODE>: <LEVEL>}
  \item \texttt{pod-security.kubernetes.io/<MODE>-version: <VERSION>}
  \item Enforce, audit, warn
 \end{itemize}

\end{frame}


\begin{frame}{root == root != root}
 \begin{itemize}
  \item Capabilities
  % demo: iptables
  \item hostUsers
 \end{itemize}
\end{frame}

\begin{frame}{Privileged}
 \begin{itemize}
  \item 
 \end{itemize}
\end{frame}

% TASK: Make deployment follow Pod security standard - restricted 

\begin{frame}{TASK: Create a hardened deployment}
 \begin{itemize}
  \item Make a deployment of \texttt{ghcr.io/scilifelabdatacentre/menu-backend:latest}
  \item Create a new namespace and apply the pod security standard restricted to it
  \item Update your deployment to allow deployment in the created namespace
 \end{itemize}
\end{frame}


\begin{frame}
 \begin{itemize}
  \item Run containers using hardened runtimes
  \item Kata containers
  \item gVisor
 \end{itemize}
\end{frame}
% TASK: https://killercoda.com/killer-shell-cks/scenario/sandbox-gvisor
 
 \begin{frame}{Compliance}
  \begin{itemize}
   \item Not just a checklist
   \item Aid to make your systems more secure 
  \end{itemize}
 \end{frame}

 
 \begin{frame}
 Container Drift --- Neuvector, Aqua?
 \end{frame}
 % DEMO: Neuvector
 
\end{document}
